%\documentclass[aps,onecolumn,preprint,superscriptaddress,nofootinbib,floats]{revtex4}
%\usepackage{graphicx}

\documentclass[12pt]{article}

\def\beq{\begin{equation}}
\def\eeq{\end{equation}}


\usepackage{amsmath,amssymb,graphicx,multirow,xspace,slashed}
\usepackage[colorlinks=true,urlcolor=blue,anchorcolor=blue,citecolor=blue,filecolor=blue,linkcolor=blue,menucolor=blue,pagecolor=blue]{hyperref}


\usepackage{floatrow}
% Table float box with bottom caption, box width adjusted to content
\newfloatcommand{capbtabbox}{table}[][\FBwidth]


\usepackage[font=footnotesize,labelfont=bf]{caption}

\newcommand*\myat{{\fontfamily{ptm}\selectfont @}}

%\usepackage[notref]{showkeys}
\usepackage{lineno}

\allowdisplaybreaks
\bibliographystyle{JHEP}

\addtolength{\oddsidemargin}{-.4in}
\addtolength{\evensidemargin}{-.4in}
\addtolength{\textwidth}{0.8in}
\addtolength{\topmargin}{-.6in}
\addtolength{\textheight}{1in}
%\addtolength{\footskip}{0.3in}
\renewcommand{\baselinestretch}{1.2}

\long\def\symbolfootnote[#1]#2{\begingroup%
\def\thefootnote{\fnsymbol{footnote}}\footnote[#1]{#2}\endgroup}

\renewcommand{\textfraction}{0}
\renewcommand{\topfraction}{0.95}


\newcommand{\newc}{\newcommand}
\newc{\gsim}{\lower.7ex\hbox{$\;\stackrel{\textstyle>}{\sim}\;$}}
\newc{\lsim}{\lower.7ex\hbox{$\;\stackrel{\textstyle<}{\sim}\;$}}
\newc{\gev}{\,{\rm GeV}}
\newc{\mev}{\,{\rm MeV}}
\newc{\ev}{\,{\rm eV}}
\newc{\kev}{\,{\rm keV}}
\newc{\tev}{\,{\rm TeV}}

\newcommand{\ifb}{\,\mathrm{fb}^{-1}}
\newcommand{\ipb}{\,\mathrm{pb}^{-1}}
\renewcommand*\descriptionlabel[1]{\hspace\labelsep\normalfont #1}

\def\ln{\mathop{\rm ln}}
\def\tr{\mathop{\rm tr}}
\def\Tr{\mathop{\rm Tr}}
\def\Im{\mathop{\rm Im}}
\def\Re{\mathop{\rm Re}}
\def\bR{\mathop{\bf R}}
\def\bC{\mathop{\bf C}}
\def\lie{\mathop{\hbox{\it\$}}} %pound sterling
\newc{\mz}{M_Z}
\newc{\mpl}{M_*}
\newc{\mw}{m_{\rm weak}}
\newc{\nr}[1]{N^c_R{}_{#1}}

%\renewcommand{\phi}{\varphi}

%indices and other greek stuff
\renewcommand{\a}{\alpha}
\newcommand{\da}{{\dot \alpha}}
\renewcommand{\b}{\beta}
\newcommand{\db}{{\dot\beta}}
\newcommand{\g}{\gamma}
\newcommand{\dg}{{\dot\gamma}}
\renewcommand{\d}{\delta}
\newcommand{\dd}{{\dot\delta}}
\newcommand{\m}{\mu}
\newcommand{\n}{\nu}
\newcommand{\e}{\epsilon}
\newcommand{\s}{\sigma} 
\renewcommand{\r}{\rho}
\newcommand{\bs}{{\bar\sigma}}
\renewcommand{\l}{\lambda}
\renewcommand{\L}{\Lambda}
\renewcommand{\k}{\kappa}
\renewcommand{\th}{\theta}
\newcommand{\thb}{{\bar\theta}}
\newcommand{\D}{\Delta}
\newcommand{\B}{\bar B_\mu}
\newcommand{\cA}{c_{A_{u,d}}}
\newcommand{\cH}{c_{m_{u,d}}}
\renewcommand{\dag}{\dagger}
\newcommand{\bra}{\langle}
\newcommand{\ket}{\rangle}
\newcommand{\Q}{\bar Q}
\renewcommand{\O}{O}

\newcommand{\CM}{{\mathcal M}}

%%%%%%%%%%%%%%%%%%%%%%%% special abrev's %%%%%%%%%%%%%%%%%%%%%%%%%%%%%

\newcommand{\mhu}{{\hat m_{H_u}}}
\newcommand{\mhd}{{\hat m_{H_d}}}
\newcommand{\mhud}{{\hat m_{H_{u,d}}}}


%%%%%%%%%%%%%%%%%%%%%%% latex eqn abrev's %%%%%%%%%%%%%%%%%%%%%%%%%%%%

\def\beq{\begin{equation}}
\def\eeq{\end{equation}}
\newcommand{\bea}{\begin{eqnarray}\begin{aligned}}
\newcommand{\eea}{\end{aligned}\end{eqnarray}}
\def\bitem{\begin{itemize}}
\def\eitem{\end{itemize}}
%
%
%%%%%%%%%%%%%%%%%%%%%%% common abrev's %%%%%%%%%%%%%%%%%
%
%

\newc{\ie}{{\it i.e.}}          \newc{\etal}{{\it et al.}}
\newc{\eg}{{\it e.g.}}          \newc{\etc}{{\it etc.}}
\newc{\cf}{{\it c.f.}}
\newcommand{\kahler}{K\"{a}hler }

\newcommand{\lang}{\mathcal{L}}
\newcommand{\C}{\mathbb{C}}
\newcommand{\CO}{O}
\newcommand{\half}{\frac{1}{2}}


%\renewcommand{\epvar}{\varepsilon}
%\renewcommand{\phi}{\varphi}
\renewcommand{\topfraction}{0.85}
\renewcommand{\textfraction}{0.1}
\renewcommand{\floatpagefraction}{0.75}

%number equations by section
 %\numberwithin{equation}{section}

%toc depth
\setcounter{tocdepth}{2}

%Begin special definitions for Instructions file
\newcommand{\ttbs}{\char'134}%\backslash for \tt
\newcommand\fverb{\setbox\fverbbox=\hbox\bgroup\verb}
\newcommand\fverbdo{\egroup\medskip\noindent%
            \fbox{\unhbox\fverbbox}\ }
\newcommand\fverbit{\egroup\item[\fbox{\unhbox\fverbbox}]}
\newbox\fverbbox
\newcommand{\jhepname}{JHEP}
%end


\renewcommand{\arraystretch}{1.3}

%\usepackage[usenames,dvipsnames]{xcolor}


\usepackage{tikz}

\newcommand{\shrug}[1][]{%
\begin{tikzpicture}[baseline,x=0.8\ht\strutbox,y=0.8\ht\strutbox,line width=0.125ex,#1]
\def\arm{(-2.5,0.95) to (-2,0.95) (-1.9,1) to (-1.5,0) (-1.35,0) to (-0.8,0)};
\draw \arm;
\draw[xscale=-1] \arm;
\def\headpart{(0.6,0) arc[start angle=-40, end angle=40,x radius=0.6,y radius=0.8]};
\draw \headpart;
\draw[xscale=-1] \headpart;
\def\eye{(-0.075,0.15) .. controls (0.02,0) .. (0.075,-0.15)};
\draw[shift={(-0.3,0.8)}] \eye;
\draw[shift={(0,0.85)}] \eye;
% draw mouth
\draw (-0.1,0.2) to [out=15,in=-100] (0.4,0.95); 
\end{tikzpicture}}



\newcommand{\MYhref}[3][blue]{\href{#2}{\color{#1}{#3}}}%

\begin{document}

\title{Toy Generative Model for Jets}
\maketitle


In this notes, first we study the requirements for the simplest toy generative model that would resemble a full generative model for jets. Next, we describe how to build such a model.

\section{Kinematic considerations}

Let start by considering a a 2-body decay. In the parent rest frame, we have the parent momentum $p^\mu_p=p^\mu_{\text{L}}+p^\mu_{\text{R}}=(\sqrt{s}, 0, 0, 0)$. From requiring 4-momentum conservation, the children energies are given by
\bea
E_{\text{L}}=\frac{\sqrt{s}}{2}\bigg(1+\frac{m_{\text{L}}^2}{s}-\frac{m_{\text{R}}^2}{s} \bigg) \\
E_{\text{R}}=\frac{\sqrt{s}}{2}\bigg(1+\frac{m_{\text{R}}^2}{s}-\frac{m_{\text{L}}^2}{s} \bigg)
\eea

and the magnitude of their 3-momentum by
\bea\label{eq:Prestframe}
|\vec{p}| =\frac{\sqrt{s}}{2} \bar{\beta}=\frac{\sqrt{s}}{2} \sqrt{1-\frac{2 (m_{\text{L}}^2+m_{\text{R}}^2)}{s}+\frac{(m_{\text{L}}^2-m_{\text{R}}^2)^2}{s^2}}
\eea



%%------------------------------------------------------------------------------------
\subsection{Simplest case:  $m_{\text{L}}=m_{\text{R}}=m$}
In this section we will specialize to the case where $m_{\text{L}}=m_{\text{R}}=m$ (we will comment on the general case in the next subsection).
\bea
E_{\text{L}}=E_{\text{R}}=\frac{\sqrt{s}}{2} = E \\
|\vec{p}| =E \sqrt{1-\frac{m^2}{E^2}}
\eea


If we  apply a boost to the lab frame, with factor $\beta \hat{n}$, we obtain for the children 3-momentum
\bea\label{eq:pChildLab}
\vec{p}^{\it{\,l}}_{\text{L}} = - \gamma E \beta \hat{n} \,-\,  E \sqrt{1-\frac{m^2}{E^2}} \,\,\,[ \hat{r} + (\gamma -1) (\hat{r} \cdot \hat{n}) \hat{n} ] \\
\vec{p}^{\it{\,l}}_{\text{R}}= - \gamma E \beta \hat{n} \,+\,  E \sqrt{1-\frac{m^2}{E^2}} \,\,\,[ \hat{r} + (\gamma -1) (\hat{r} \cdot \hat{n}) \hat{n} ]
\eea
with $\hat{r}$ a unit vector that specifies the direction of the children momentum in the parent rest frame. For the parent we have 
\bea\label{eq:pParentLab}
\vec{p}^{\it{\,l}}_p= - 2 \gamma E \beta \hat{n} 
\eea

From (\ref{eq:pParentLab}), we can rewrite (\ref{eq:pChildLab}) as:
\bea\label{eq:pChildLab2}
\vec{p}^{\it{\,l}}_{\text{L}}= \frac{1}{2} \vec{p}^{\it{\,l}}_p - \vec{\Delta}  \\
\vec{p}^{\it{\,l}}_{\text{R}}= \frac{1}{2} \vec{p}^{\it{\,l}}_p + \vec{\Delta}
\eea
with
\bea\label{eq:delta}
 \vec{\Delta}   = E \sqrt{1-\frac{m^2}{E^2}} \,\,\,[ \hat{r} + (\gamma -1) (\hat{r} \cdot \hat{n}) \hat{n} ] 
\eea

Let's study the dependence of $|\vec{\Delta}|=\Delta(E,m,\hat{r},\gamma,\hat{n})$. We will work in the (y,z) plane.

\begin{itemize}
\item $\gamma=\frac{E_p}{m_p}$ or $\gamma \beta = |\vec{p}_p|/m_p$
\item If we draw an angle $\phi \in \{-\pi,\pi\}$ from a uniform distribution, we get in the $(\hat{y},\hat{z})$ plane,  $\hat{r}= (\sin{\phi}, \cos{\phi})$.

\item At each step, we draw $\phi$ and $m$, and boost the children to the lab frame. Next, we promote each child to a parent and repeat the process. Thus, the unit vector $\hat{n}$ is fixed by the previous step and given by $\hat{n}=(\sin{\theta_\text{p}}, \cos{\theta_\text{p}})$ with $\theta_\text{p}=\tan^{-1}\bigg({\frac{(p_{\text{p}})_y}{(p_{\text{p}})_z}}\bigg)$ the parent angle with respect to the $\hat{z}$ axis in the lab frame. 

\item At each step, $E_{\text{L}}=E_{\text{R}}=E =\frac{\sqrt{s}}{2}= \frac{m_{\text{p}}}{2}$ is fixed by the parent mass.

\end{itemize}

As a result, we obtain $|\Delta|=\Delta(m,\phi)$ given by
\bea
 \Delta_y   &= \frac{m_\text{p}}{2} \sqrt{1- 4\frac{m^2}{m_\text{p}^2}}  \bigg[ \sin{\phi} + \bigg(\frac{m_\text{p}}{ 2 m}-1\bigg) \cos{(\phi-\theta_\text{p})} \sin{\theta_\text{p}} \bigg]\\ 
  \Delta_z   &= \frac{m_\text{p}}{2} \sqrt{1- 4\frac{m^2}{m_\text{p}^2}}   \bigg[\cos{\phi} +  \bigg(\frac{m_\text{p}}{2 m}-1\bigg) \cos{(\phi-\theta_\text{p})} \cos{\theta_\text{p}} \bigg]
\eea


%%---------------------------------------------------------------------------------------
\subsection{General case, with  $m_{\text{L}}\neq m_{\text{R}}$}

In this case, (\ref{eq:pChildLab2}) becomes
\bea\label{eq:pChildLab2Gen}
\vec{p}^{\,\it{l}}_{\text{L}}= \frac{E_{\text{L}}}{E_\text{p}} \vec{p}^{\,\it{l}}_\text{p} - |\vec{p}| \,\,\,[ \hat{r} + (\gamma -1) (\hat{r} \cdot \hat{n}) \hat{n} ]  \\
\vec{p}^{\,\it{l}}_{\text{R}}=  \frac{E_{\text{R}}}{E_\text{p}} \vec{p}^{\,\it{l}}_\text{p} +  |\vec{p}| \,\,\,[ \hat{r} + (\gamma -1) (\hat{r} \cdot \hat{n}) \hat{n} ] 
\eea
where $|\vec{p}|$ is as in (\ref{eq:Prestframe}):
\bea
|\vec{p}| =\frac{\sqrt{s}}{2} \bar{\beta}=\frac{\sqrt{s}}{2} \sqrt{1-\frac{2 (m_{\text{L}}^2+m_{\text{R}}^2)}{s}+\frac{(m_{\text{L}}^2-m_{\text{R}}^2)^2}{s^2}}\nonumber
\eea

% =====================================================
\vspace{0.6cm}
\section{Toy Model for Jets}

In this section we describe the model we built as well as the previous versions and their issues. We also mention other possibilities for a toy model for jets.


\subsection{Minimum requirements}

The traditional clustering algorithms are based on a measure given by:
\bea\label{eq:dij}
d_{ij}= \text{min}({p_{\text{T}i}^{2\alpha}, {p_{\text{T}j}^{2\alpha}})\,\, \frac{\Delta R_{ij}^2}{R^2}}
\eea
where $\Delta R_{ij}=(\phi_i-\phi_j)$ in a 2D model, with $\phi_i$ the angle of particle $i$ with respect to the $\hat{z}$ axis in the lab frame. Also, $\alpha=\{-1,0,1\}$ defines the $\{\text{anti-kt, CA and kt}\}$ algorithms respectively. Thus for a meaningful definition of $d_{ij}$ the toy model should be at least 2D. 


%%---------------------------------------------------------------------------------------
\subsection{Current model: Model 4 }

We build a 2D model in the $(y,z)$ plane, where $\hat{z}$ is the direction of the beam axis and $\hat{y}$ the transverse direction. 
We define $p_{\text{T}} =|p_y|$.

%\vspace{0.6cm}
%\hspace{-0.8cm}
%%-------------
\subsubsection{Splitting of a node}

Given a parent node with momentum $\vec{p}_\text{p}$, and a scalar value $\Delta_{\text{p}}$ that sets the scale of the splitting, we define the splitting function as follows:
\begin{enumerate}

\item We draw a value $\phi_\text{p}$ for the angle in the (y,z) plane, from a uniform distribution in the range $\{0,2\pi\}$ and define
\bea\label{eq:Deltavec}
\vec{\Delta}_\text{p}= \Delta_\text{p}\,\,(\sin\phi_\text{p},\cos\phi_\text{p})
\eea 

\item We define the L,R children nodes momentum following (\ref{eq:pChildLab2}),
\bea\label{eq:pLR}
\vec{p}_\text{L}= \frac{1}{2} \vec{p}_\text{p} - \vec{\Delta}_\text{p}  \\
\vec{p}_\text{R}= \frac{1}{2} \vec{p}_\text{p} +\vec{\Delta}_\text{p}
\eea

\item We separately draw $r_\text{L}$ and $r_\text{R}$ from an exponential distribution 
\bea \label{eq:exponential}
f(x,\lambda)=\lambda e^{-\lambda x} 
\eea

and generate
\bea
\Delta_\text{L} &= \Delta_\text{p} \,\, r_\text{L}\\
\Delta_\text{R} &= \Delta_\text{p} \,\, r_\text{R}
\eea

\end{enumerate}

%%-------------
\subsubsection{Generative process}

Input parameters:
\begin{itemize}

\item $\vec{p}_0$: momentum of the jet. This will be the input value for the root node of the tree.
\item $\lambda$: decaying rate for the exponential distribution.
\item $\Delta_0$: Initial scale (units of energy). 
\item $\Delta_\text{cut}$: cut-off scale to stop the showering process (units of energy). 

\end{itemize}

We start with the root of the tree as the parent node and build the jet binary tree recursively as follows:
\begin{itemize}

\item We split the parent node, and get $\vec{p}_\text{L}$, $\vec{p}_\text{R}$, $\Delta_\text{L}$, $\Delta_\text{R}$.

\item If $\Delta_\text{L/R} > \Delta_\text{cut}$, we promote the L/R node to a parent node (i.e. we promote $\vec{p}_\text{L/R}$ to $\vec{p}_\text{p}$ and $\Delta_\text{L/R}$ to $\Delta_\text{p}$), and split again.

\end{itemize}

%%-------------
\subsubsection{Heavy resonance vs QCD like jet}

To model a jet coming from a heavy resonance X decay, e.g. a W boson jet, we use $\Delta_{\text{p}} =  \frac{m_X}{2}$ to split the root node.

To model a QCD like jet, we split the root node with $\Delta_{\text{p}} = \Delta_0 \,\, r_{\text{root}}$, where $r_{\text{root}}$ is drawn from the exponential distribution (\ref{eq:exponential}).

%%-------------
\subsubsection{Reconstruct $\{\vec{p}_\text{p}$, $\Delta_{\text{p}}$, $\phi_{\text{p}}$, $r_L$, $r_R\}$ from  $\{\Delta_L$, $\Delta_R$, $\vec{p}_\text{L}$, $ \vec{p}_\text{R}\}$ }

We show how to reconstruct $\{\vec{p}_\text{p}$, $\Delta_{\text{p}}$, $\phi_{\text{p}}$, $r_L$, $r_R\}$ from a bottom-up approach for each splitting. From  $\{\Delta_L$, $\Delta_R$, $\vec{p}_\text{L}$, $ \vec{p}_\text{R}\}$ we get:
\bea
\vec{p}_\text{p} &= \vec{p}_\text{L}+ \vec{p}_\text{R}\\
\vec{\Delta}_\text{p} &= \frac{1}{2} (\vec{p}_\text{R} - \vec{p}_\text{L})
\eea
Next, we get
\bea
\Delta_\text{p} &= | \vec{ \Delta}_\text{p} |\\
\phi_{\text{p}} &=\tan^{-1}\frac{(\Delta_\text{p})_y}{(\Delta_\text{p})_z}\\
r_{\text{L}} &=\frac{\Delta_{\text{L}}}{\Delta_{\text{p}}}\\
r_{\text{R}} &=\frac{\Delta_{\text{R}}}{\Delta_{\text{p}}}
\eea
If the L/R node is a leaf, then we set
\bea
\Delta_{\text{L/R}}=\Delta_{\text{cut}}
\eea


%%---------------------------------------------------------------------------------------
\subsection{Previous toy models and their issues}


\subsubsection{Model 1}

This was a 1D model where $d_{ij}$ in (\ref{eq:dij}) was not well defined.

%%-------------
\subsubsection{Model 2}

In this model we defined the splitting of (\ref{eq:pChildLab2}) as
\bea\label{eq:pLRb}
\vec{p}_\text{L}= \vec{p}_\text{p} - \vec{\Delta}_\text{p}  \\
\vec{p}_\text{R}= \vec{p}_\text{p} +\vec{\Delta}_\text{p}
\eea
which gives
\bea
\vec{p}_\text{p}= \frac{1}{2} (\vec{p}_\text{L} + \vec{p}_\text{R} ) 
\eea
As a result, we would get jets with different momentum $\vec{p}$ depending on the clustering algorithm, e.g. $k_t$, $\text{anti-}k_t$, etc. The reason is that the jet momentum in this case is not permutation invariant with respect to the order in which we cluster its constituents.\footnote{E.g. when clustering 3 jet constituents with momentum $\{a,b,c\}$ we see $\frac{1}{2}(\frac{a+b}{2})+\frac{c}{2} \neq \frac{1}{2}(\frac{a+c}{2})+\frac{b}{2}$}

%%-------------
\subsubsection{Model 3}

This model differs from Model 4 in the way we do the splitting of a node, which also changes the generative process. In this case, given a parent node with momentum $\vec{p}_\text{p}$, and a scalar value $\Delta_{\text{gp}}$ that comes from the grandparent node, we define the splitting function as follows:
\begin{enumerate}

\item We draw $r_\text{p}$ from the exponential distribution of (\ref{eq:exponential}) and calculate
\bea\label{rp}
\Delta_\text{p} &= \Delta_\text{gp} \,\, r_\text{p}
\eea

\item We draw a value $\phi_\text{p}$ for the angle in the (y,z) plane, from a uniform distribution in the range $\{0,2\pi\}$ and define as in (\ref{eq:Deltavec})
\bea
\vec{\Delta}_\text{p}= \Delta_\text{p}\,\,(\sin\phi_\text{p},\cos\phi_\text{p})\nonumber
\eea 

\item If $\Delta_\text{p} > \Delta_\text{cut}$ we split the parent node by defining the L,R children nodes momentum following (\ref{eq:pChildLab2}),
\bea
\vec{p}_\text{L}= \frac{1}{2} \vec{p}_\text{p} - \vec{\Delta}_\text{p}  \\
\vec{p}_\text{R}= \frac{1}{2} \vec{p}_\text{p} +\vec{\Delta}_\text{p} \nonumber
\eea

\end{enumerate}

Next, we promote the L and R nodes to a parent node each (i.e. we promote $\vec{p}_\text{L/R}$ to $\vec{p}_\text{p}$ and $\Delta_\text{p}$ to $\Delta_\text{gp}$), and repeat the splitting process.


The problem with this model is that we cannot reconstruct $\Delta_\text{gp}$ from a bottom-up approach. As a result, we cannot derive from (\ref{rp}) the value $r_\text{p}$  drawn from the distribution. 


%%---------------------------------------------------------------------------------------
\subsection{Other ideas for Toy Generative Models}

We could define $\vec{\Delta}$  as in (\ref{eq:delta}):
\bea
 \vec{\Delta}   = E \sqrt{1-\frac{m^2}{E^2}} \,\,\,[ \hat{r} + (\gamma -1) (\hat{r} \cdot \hat{n}) \hat{n} ] = \vec{\Delta}(m,\phi)
\eea

We draw $\phi$ from a uniform distribution. We can get $m$ in a way to resemble the sudakov factor approach as:
\beq\label{eq:drawMass}
m_{\text{L/R}} = \frac{m_{\text{p}}}{2}\,\,  r_{\text{L/R}} 
\eeq
where $r_{\text{L/R}}$ is drawn from an exponential distribution $f(x,\lambda) \propto e^{-\lambda x}$ for $x \in (0,1)$. The prescription of (\ref{eq:drawMass}) solves one of the problems of the traditional parton showers given that it satisfies $m_{\text{L}} + m_{\text{R}} \leq m_{\text{p}}$\footnote{Traditional parton showers draw $m_{\text{L}}$ and $m_{\text{R}}$ independently, so they should check the constraint is satisfied.}. Then, we could think of $m_{\text{L/R}}$ as the off-shell mass value, to avoid the required reshuffling. This results in leaves where each pair of siblings have the same mass, all different among pairs of siblings.


This case would be closer to a physics parton shower but more complex and time consuming.

%%-------------
\subsubsection{General case, with  $m_{\text{L}}\neq m_{\text{R}}$}

We could also build a model for this case, adding extra features. 
We should replace (\ref{eq:drawMass}) by
\bea
m_{\text{L}} &= m_{\text{p}}\,\,  r_{\text{L}}  \\
m_{\text{R}} &= (m_{\text{p}}-m_{\text{L}})\,\,  r_{\text{R}}
\eea
where $r_{\text{L}}$ and $r_{\text{R}}$ are independently drawn from an exponential distribution $f(x,\lambda) \propto e^{-\lambda x}$ for $x \in (0,1)$.


\end{document}
